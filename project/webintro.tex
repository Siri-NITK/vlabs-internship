% Created 2016-05-17 Tue 11:02
\documentclass[11pt]{article}
\usepackage[latin1]{inputenc}
\usepackage[T1]{fontenc}
\usepackage{fixltx2e}
\usepackage{graphicx}
\usepackage{longtable}
\usepackage{float}
\usepackage{wrapfig}
\usepackage{soul}
\usepackage{textcomp}
\usepackage{marvosym}
\usepackage{wasysym}
\usepackage{latexsym}
\usepackage{amssymb}
\usepackage{hyperref}
\tolerance=1000
\providecommand{\alert}[1]{\textbf{#1}}

\title{Building a Web Application:  Ground up to the cloud}
\author{Srivalya Elluru}
\date{2016-05-17}
\hypersetup{
  pdfkeywords={},
  pdfsubject={},
  pdfcreator={Emacs Org-mode version 7.9.3f}}

\begin{document}

\maketitle

\setcounter{tocdepth}{3}
\tableofcontents
\vspace*{1cm}

\section{Overview}
\label{sec-1}
\subsection{what is a WebApplication}
\label{sec-1-1}

 It is client-server application which is stored on remote server.
 There are interfaces such as user interface,web interface etc.
 
  
\section{Introduction}
\label{sec-2}
\subsection{The components of of Web Application}
\label{sec-2-1}

   +Client Side
   +Server Side
   +Database 
   +Data Model
   The client send a request to the web server through the browser
   interface as http request.
\section{SDLC}
\label{sec-3}


  -\href{http://python.org}{python} ::A pytho web page    
  -\href{http://flask.pocoo.org/}{flask} :: Its a microframework for python

  -\href{file:///home/home/Downloads/IMG_20160517_092640205.jpg}{Imageofwebapplication}

\end{document}
